\documentclass[a4paper,onecolumn,12pt]{revtex4-1}

\usepackage{fancyhdr}
% \usepackage{romannum}
\usepackage{graphicx}         % include graphics such as plots
\usepackage{hyperref}
\usepackage[dvipsnames]{xcolor}
\usepackage{physics,amssymb}  % mathematical symbols (physics imports amsmath)
\usepackage{slashed}          % for Feynman slash notation

% \usepackage[numbers,sort&compress]{natbib}
\hypersetup{
    colorlinks=true,
    linkcolor=blue,
    citecolor=red,
    filecolor=red,
    urlcolor=blue,
}


\newcommand\numberthis{\addtocounter{equation}{1}\tag{\theequation}}
\newcommand{\ihat}{\boldsymbol{\hat{\textbf{\i}}}}
\newcommand{\jhat}{\boldsymbol{\hat{\textbf{\j}}}}
\newcommand{\khat}{\boldsymbol{\hat{\textbf{k}}}}
\newcommand{\vc}[1]{\mathbf{#1}}

\graphicspath{{../plots/}} % search for figures in this dir

\pagestyle{fancy}
\lhead{FIRAS}
\rhead{Vegard Falmår}

%\evensidemargin=-0.1in
%\oddsidemargin=-0.1in
\setlength{\textwidth}{420pt}
\setlength{\headheight}{14.0pt}

\begin{document}

\begin{center}
{\bf \Large CMB Temperature From FIRAS Data\\}
{\bf Vegard Falmår\\}
\today
\end{center}





\section{Task}

Analyse the
\href{https://lambda.gsfc.nasa.gov/data/cobe/firas/monopole_spec/firas_monopole_spec_v1.txt}{FIRAS data set}.
It is the average spectrum of the sky as measured by the FIRAS experiment, which is considered to be a near-perfect blackbody, given by the Planck blackbody law.

Estimate the temperature of the blackbody and the uncertainty of the temperature. One important thing to note is the non-standard frequency units, reported in inverse-centimetres.

We also ask that you perform the analysis using as few external packages as possible, for example avoiding the use of \texttt{scipy.minimize} or other
``black box'' routines.





\section{Data Set}

The data set contains five sets of data points.

\subsection{Frequency}

The frequency is given in units of inverse-centimetres (cm\(^{-1}\)). If desired this can be converted to Hertz through multiplying with \(100 c\), where \(c \approx 3 \cdot 10^8\) m/s is the speed of light.



\subsection{Spectrum}

Intensity, as obtained from Planck's blackbody law, given in units of MegaJanskys per steradian (MJy / sr).





\section{Planck's law of blackbody radiation}

Reads
\begin{equation}
  B(\nu, T)
    = \frac{2 h \nu^3}{c^2} \frac{1}{e^{h \nu / (k_B T)} - 1}.
\end{equation}
In SI units it is expressed as
\begin{equation}
  \text{J} \;\text{m}^{-2} \;\text{sr}^{-1}
    = \text{W} \; \text{m}^{-2} \; \text{sr}^{-1} \; \text{Hz}^{-1}.
\end{equation}
To convert into MJy / sr multiply with
\begin{itemize}
  \item \(10^{26}\) for the Janskys (\(1 \text{Jy} = 10^{-26} \text{W} \; \text{m}^{-2} \;\text{Hz}^{-1}\)) and
  \item \(10^{-6}\) for the Mega.
\end{itemize}




\end{document}
